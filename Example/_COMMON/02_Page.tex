
\begin{flushleft}
    {\setmainfont[Path=FONT/]{KOPUBWORLD_DOTUM_PRO_BOLD.OTF}\textcolor{skyblue2}{{\huge\textbf{2.}}}}
    {\setmainhangulfont[Path=FONT/]{KOPUBWORLD_DOTUM_PRO_BOLD.OTF}\textcolor{skyblue2}{{\huge\textbf{닮음비와 넓이 $\cdot$ 부피와 비의 관계}}}}

\end{flushleft}

\begin{flushleft}
    칠각형의 한 꼭짓점에서 그을 수 있는 대각선의 개수는 4개이며, 
    이 대각선으로 \\ 5개의 삼각형이 만들어진다.
    이때, 삼각형의 세 내각의 크기의 합은 $180^{\circ}$이므로
    칠각형의 내각의 크기의 합은 $900^{\circ}$임을 알 수 있다. 
\end{flushleft}

\begin{flushleft}
    $(1)$ $n$각형의 내각의 크기의 합은 $180^{\circ} \times (n-2)$개이다.
\end{flushleft}

\begin{tcolorbox}[colback = white, colframe = blue!35!skyblue, title = \textmd{이해하기}]
    $n$각형의 한 꼭짓점에서 그을 수 있는 대각선은 $(n-3)$개이므로
    $n$개의 꼭짓점에서 그을 수 있는 대각선은 $n(n-3)$개이다.
    \,그러나 한 대각선 위에는 $2$개의 꼭짓점이 있으므로 $n$각형의 대각선은
    $\frac{n(n-3)}{2}$개이다.
\end{tcolorbox}

\begin{flushleft}
    $(2)$ $n$각형의 내각의 크기의 합은 $180^{\circ} \times (n-2)$개이다.
\end{flushleft}



\begin{flushleft}
    {\setmainfont[Path=FONT/]{KOPUBWORLD_DOTUM_PRO_BOLD.OTF}\textcolor{skyblue2}{{\huge\textbf{3.}}}}
    {\setmainhangulfont[Path=FONT/]{KOPUBWORLD_DOTUM_PRO_BOLD.OTF}\textcolor{skyblue2}{{\huge\textbf{평행선과 넓이}}}}
\end{flushleft}

\begin{flushleft}
    $(3)$ 삼각형의 한 외각의 크기는 그와 이웃하지 않는 두 내각의 크기의 합과 같다.
\end{flushleft}

\begin{tcolorbox}[colback = white, colframe = blue!35!skyblue, title = \textmd{이해하기}]
    $\overline{AB} \parallel \overline{EC}$일 때, $\angle A = \angle ACE$ (엇각) $\angle B = \angle ECD$(동위각)이므로\\ 
    $\angle A + \angle B + \angle C = \angle ACE + \angle ECD + \angle BCA = 180^{\circ}$ \\
    따라서, $\angle C$의 외각 $\angle ACD$의 크기는 $\angle ACD = \angle ACE + \angle ECD = \angle A + \angle B$
\end{tcolorbox}


\begin{flushleft}
    {\setmainfont[Path=FONT/]{KOPUBWORLD_DOTUM_PRO_BOLD.OTF}\textcolor{skyblue2}{{\huge\textbf{4.}}}}
    {\setmainhangulfont[Path=FONT/]{KOPUBWORLD_DOTUM_PRO_BOLD.OTF}\textcolor{skyblue2}{{\huge\textbf{프로그래밍 실습}}}}
\end{flushleft}

\begin{flushleft}
    $(4)$ \texttt{C++} 언어로 Hello World를 출력해보도록 하겠습니다.
\end{flushleft}

\begin{lstlisting}
// HelloWorld.cpp
#include <iostream>

using namespace std;

int main() {
    cout << "Hello World!!" << endl;
    return 0;
}
\end{lstlisting}