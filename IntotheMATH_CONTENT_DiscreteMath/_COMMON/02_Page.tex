%%%%%%%%%%%%%%%%%%%%%%%%% 상단 제목 %%%%%%%%%%%%%%%%%%%%%%%%%
\begin{tikzpicture}[remember picture,overlay]
    \draw[fill = blue!40] (0.46,0) rectangle (25, 4); 
    \node[draw = blue!40] at (3, 1.7)  {\setmainhangulfont[Path=FONT/]{KOPUBWORLD_DOTUM_PRO_BOLD.OTF}{\textcolor{white}{{\huge 이산수학 개론}}}};  
    \node[draw = blue!40] at (5.3, 0.8)  {\setmainhangulfont[Path=FONT/]{KOPUBWORLD_DOTUM_PRO_MEDIUM.OTF}{\textcolor{white}{{\small 행과 열을 수학적으로 표기하는 기법인 행렬 대해서 알아봅시다.}}}};
    % (x축, y축)
\end{tikzpicture}

\begin{tikzpicture}[remember picture,overlay]
    \draw[fill = black!90] (-2.51, 3.4) rectangle (0.45, 0.55); 
    \node[] at (-1, 2.33) {\setmainfont[Path=FONT/]{KOPUBWORLD_BATANG_PRO_MEDIUM.OTF}{\textcolor{skyblue}{{\Huge {Alpha}}}}};
    \node[] at (-1, 1.45)  {\setmainfont[Path=FONT/]{KOPUBWORLD_BATANG_PRO_BOLD.OTF}{{\textcolor{skyblue}{{\Huge\textbf {1.2}}}}}};
\end{tikzpicture}

%%%%%%%%%%%%%%%%%%%%%%%%%%%%%%%%%%%%%%%%%%%%%%%%%% 행렬 %%%%%%%%%%%%%%%%%%%%%%%%%%%%%%%%%%%%%%%%%%%%%%%%%%

\begin{flushleft}
    {\setmainfont[Path=FONT/]{KOPUBWORLD_DOTUM_PRO_BOLD.OTF} {\textcolor{header}{{\huge\textbf{1.}}}}}
    {\setmainhangulfont[Path=FONT/]{KOPUBWORLD_DOTUM_PRO_BOLD.OTF} {\textcolor{header}{{\huge\textbf{행렬}}}}}
\end{flushleft}

\begin{flushleft}
    행렬은 행과 열로 나열하는 것을 말합니다. 기본적으로 연립방정식을 풀기 위하여 만들어진 개념이며, 수, 문자, 함수 등을 괄호 안에 배열한 것입니다. 행렬의 각 성분은 실수여야 하고, 이는 “스칼라(Scalar)”라고 합니다. 스칼라는 크기만 있고 방향을 가지지 않는 양을 말하며, 벡터와는 반대되는 개념입니다. 수학적으로 정의한 행렬은 다음과 같습니다. 
\end{flushleft}

\begin{tcolorbox}[colback = white, colframe = Definition, title = \textmd{정의: 행렬}]
    행렬 $A$는 실수들을 사각형의 배열로 표시한 것이다(단, $m$과 $n$은 양의 정수). 각각 $n$쌍으로 된 $m$개의 수평 성분 ($a_{i1} \ a_{i2} \ a_{i3} \cdots  a_{in} $) (단, $1 \leq i \leq m $)을 $A$의 행(row)이라고 하고, 각각 $m$쌍으로 된 $n$개의 수직 성분 ($a_{1j} \ a_{2j} \ a_{3j} \cdots  a_{mj} $)(단, $1 \leq j \leq n$)을 $A$의 열(column)이라고 한다.
    \[ A = \begin{pmatrix} a_{11} & a_{12} & a_{13} & \cdots & a_{1n} \\  
        a_{21} & a_{22} & a_{23} & \cdots  & a_{2n}  \\ && \vdots &&  \\
        a_{m1} & a_{m2} & a_{m3} & \cdots & a_{mn}\end{pmatrix}  \]
\end{tcolorbox}

\begin{flushleft}
행렬 $A$를 $A = (a_{ij})$로 표시하는데, 이때, 원소 $a_{ij}$는 $i$행의 $j$번째 열의 원소를 나타냅니다. $A$를 $m \times n$ 행렬이라고 하며, $m \ by \ n$ 행렬로 읽습니다.
\end{flushleft}

\bigskip
\begin{flushleft}
    {\setmainfont[Path=FONT/]{KOPUBWORLD_DOTUM_PRO_BOLD.OTF} {\textcolor{header}{{\huge\textbf{2.}}}}}
    {\setmainhangulfont[Path=FONT/]{KOPUBWORLD_DOTUM_PRO_BOLD.OTF} {\textcolor{header}{{\huge\textbf{행렬의 연산}}}}}
\end{flushleft}

\begin{flushleft}
    기초적인 행렬의 연산을 알아보기 전, "영행렬"과 "행렬의 같음 성질"에 대해서 알아야 합니다.
    먼저, 영행렬부터 알아보도록 하겠습니다.
\end{flushleft}

\bigskip
\begin{flushleft}
    {\setmainhangulfont[Path=FONT/]{KOPUBWORLD_DOTUM_PRO_BOLD.OTF}\textcolor{subheader}{{\LARGE\textbf{영행렬}}}}
\end{flushleft}

\begin{flushleft}
    만약 각 성분이 모두 0이라면, 이 행렬을 “영행렬”이라고 하고 0으로 표시합니다. 
\end{flushleft}

\begin{tcolorbox}[colback = white, colframe = Definition, title = \textmd{정의: 영행렬}]
    각 성분이 0인 행렬을 말한다. 영행렬은 $0$으로 표시한다.
    \[ \begin{pmatrix} 0 & 0 \\ 0 & 0 \end{pmatrix}, \ 
    \begin{pmatrix} 0 &0 & 0 \\ 0 & 0 & 0 \end{pmatrix}, \ 
    \begin{pmatrix} 0 & 0 & 0 \\ 0 & 0 & 0 \\ 0 & 0 & 0 \end{pmatrix} \]
\end{tcolorbox}

\bigskip
\begin{flushleft}
    {\setmainhangulfont[Path=FONT/]{KOPUBWORLD_DOTUM_PRO_BOLD.OTF}\textcolor{subheader}{{\LARGE\textbf{두 행렬의 같음}}}}
\end{flushleft}

\begin{flushleft}
    두 행렬이 있을 때, 그 두 행렬이 같을 조건은 다음과 같습니다.
\end{flushleft}

\begin{tcolorbox}[colback = white, colframe = Definition, title = \textmd{정의: 두 행렬의 같음}]
    두 행렬이 $m \times n$ 행렬이고 대응하는 원소가 모두 같으면 $A = (a_{ij})$와 $B = (b_{ij})$는 같다라고 하며, $A = B$ 라고 표현한다.
\end{tcolorbox}

\newpage

\begin{flushleft}
    지금부터는 본격적인 행렬의 연산 방식에 대해 알아보도록 하겠습니다. 행렬의 기초적인 연산으로는 행렬의 덧셈, 행렬의 실수곱, 행렬곱 등이 있습니다.
\end{flushleft}

\bigskip
\begin{flushleft}
    {\setmainhangulfont[Path=FONT/]{KOPUBWORLD_DOTUM_PRO_BOLD.OTF}\textcolor{subheader}{{\LARGE\textbf{행렬의 합}}}}
\end{flushleft}

\begin{flushleft}
    행렬의 합은 다음과 같이 정의합니다.
\end{flushleft}

\begin{tcolorbox}[colback = white, colframe = Definition, title = \textmd{정의: 행렬의 합}]
    행렬 $A$와 $B$가 같은 크기와 행과 열을 가지면, 행렬 $A$와 $B$의 행렬의 합은 $A + B$로 표시하고 $A + B = (a_{ij}) + (b_{ij}) = (a_{ij}+b_{ij})$이다. 즉, 대응하는 성분끼리 합을 구하면 그 값이 $A + B$의 각 성분이 되는 것이다.
\end{tcolorbox}

\begin{flushleft}
    행렬의 합을 통하여 알 수 있는 행렬의 덧셈 성질은 다음과 같다.
\end{flushleft}

\begin{tcolorbox}[colback = white, colframe = Theorem, title = \textmd{정리: 행렬의 덧셈 성질}]
    \begin{enumerate}
        \item $A + B = B + A$    (덧셈의 교환법칙)
        \item $(A + B) + C = A + (B + C)$    (덧셈의 결합법칙)
        \item $A + 0 = 0 + A = A$    (덧셈의 항등법칙)
    \end{enumerate}
\end{tcolorbox}

\newpage
\begin{flushleft}
    {\setmainhangulfont[Path=FONT/]{KOPUBWORLD_DOTUM_PRO_BOLD.OTF}\textcolor{subheader}{{\LARGE\textbf{스칼라 곱과 행렬의 곱}}}}
\end{flushleft}

\begin{flushleft}
    스칼라 곱은 행렬에 실수배를 하는 것입니다. 스칼라 곱은 다음과 같이 정의됩니다.
\end{flushleft}

\begin{tcolorbox}[colback = white, colframe = Definition, title = \textmd{정의: 스칼라 곱}]
    $A$가 $m \times n$ 행렬이고 $c$가 실수이면 다음이 성립한다.
    $$ c \cdot A = c \cdot (a_{ij}) = (c \times a_{ij}) $$  
\end{tcolorbox}

\begin{flushleft}
    두 행렬을 곱하는 행렬의 곱은 다음과 같이 정의됩니다.
\end{flushleft}

\begin{tcolorbox}[colback = white, colframe = Definition, title = \textmd{정의: 행렬의 곱}]
    $A = (a_{ij})$는 $m \times p $ 행렬이고 $B = (b_{ij})$는 $p \times n$ 행렬일 때, 행렬의 곱 $AB$는 $m \times n $ 행렬이고 $C = (c_{ij})$가 된다. 이때, 
    \[ C_{ij} = a_{i1}b_{1j} + a_{i2}b_{2j} \cdots + a_{ip}b_{pj} \quad (\text{단, } 1 \leq i \leq m, \ 1 \leq j \leq n) \]
    이다. 즉, 행렬의 곱 $AB$는 다음과 같이 구할 수 있다. \\

    $ AB = \begin{pmatrix} a_{11} & a_{12} & \cdots & a_{1p} \\  
    a_{21} & a_{22} & \cdots  & a_{2p}  \\ && \vdots  \\ 
    a_{i1} & a_{i2} & \cdots & a_{ip}
    \\  && \vdots   \\ 
    a_{m1} & a_{m2} & \cdots & a_{mp}\end{pmatrix} $
    $ \begin{pmatrix} b_{11} & b_{12} &  \cdots b_{1j} & \cdots & b_{1n} \\  
    b_{21} & b_{22} & \cdots b_{1j} & \cdots & b_{2n}  \\ && \vdots &&  \\ && \vdots &&  \\ && \vdots &&  \\
    b_{p1} & b_{p2} & \cdots b_{1j} & \cdots & b_{pn}\end{pmatrix} $
    =
    $ \begin{pmatrix} c_{11} & c_{12} & c_{13} & \cdots & c_{1n} \\  
    c_{21} & c_{22} & c_{23} & \cdots  & c_{2n}  \\ && \vdots &&  \\ 
    & \cdots & c_{ij} & \cdots 
    \\ && \vdots &&  \\ 
    c_{m1} & c_{m2} & c_{m3} & \cdots & c_{mn}\end{pmatrix} $
\end{tcolorbox}

\begin{flushleft}
    행렬의 곱을 통하여 알 수 있는 행렬의 곱셈 성질은 다음과 같습니다.
\end{flushleft}

\begin{tcolorbox}[colback = white, colframe = Theorem, title = \textmd{정리: 행렬의 곱셈 성질}]
    \begin{enumerate}
        \item $A \cdot (B \cdot C) = (A \cdot B) \cdot C$    (곱셈의 결함법칙)
        \item $A \cdot (B + C) = A \cdot B +A \cdot C$    (곱셈의 분배법칙)
        \item $(A + B) \cdot C = A \cdot C + B \cdot C$    (곱셈의 분배법칙)
    \end{enumerate}
\end{tcolorbox}

\newpage
\begin{flushleft}
    {\setmainfont[Path=FONT/]{KOPUBWORLD_DOTUM_PRO_BOLD.OTF} {\textcolor{header}{{\huge\textbf{3.}}}}}
    {\setmainhangulfont[Path=FONT/]{KOPUBWORLD_DOTUM_PRO_BOLD.OTF} {\textcolor{header}{{\huge\textbf{특수 행렬}}}}}
\end{flushleft}

\begin{flushleft}
    지금부터 여러 종류의 행렬들에 대해서 하나하나 알아보도록 하겠습니다. 양이 많기에 모든 행렬을 일일이 외우기보다, "이런 행렬들도 있구나" 하고 눈에 익히는 정도로만 공부하면 됩니다.
\end{flushleft}

\bigskip
\begin{flushleft}
    {\setmainhangulfont[Path=FONT/]{KOPUBWORLD_DOTUM_PRO_BOLD.OTF}\textcolor{subheader}{{\LARGE\textbf{정방 행렬}}}}
\end{flushleft}

\begin{tcolorbox}[colback = white, colframe = Definition, title = \textmd{정의: 정방 행렬}]
    행과 열의 수가 $n$으로 같은 행렬을 $n$차 정방 행렬이라고 한다. 이때, 행 또는 열의 개수를 “정방 행렬의 차수”라고 하며, $n$차 정방 행렬에서 대각선상에 위치한 원소, $a_{ii}(i = 1, \ 2, \ 3, \cdots, \ n)$를 “주대각 원소”라고 한다. 
\end{tcolorbox}

\bigskip
\begin{flushleft}
    {\setmainhangulfont[Path=FONT/]{KOPUBWORLD_DOTUM_PRO_BOLD.OTF}\textcolor{subheader}{{\LARGE\textbf{단위 행렬(항등 행렬)}}}}
\end{flushleft}

\begin{tcolorbox}[colback = white, colframe = Definition, title = \textmd{정의: 단위 행렬(항등 행렬)}]
    주대각 원소들은 모두 $1$이고, 주대각 원소를 제외한 나머지 원소들은 모두 $0$인 정방 행렬을 단위 행렬 또는 항등 행렬이라고 한다.
    \[ \begin{pmatrix} 1 & 0 \\ 0 & 1 \end{pmatrix}, \ 
    \begin{pmatrix} 1 & 0 & 0 \\ 0 & 1 & 0 \\ 0 & 0 & 1 \end{pmatrix} \]
\end{tcolorbox}

\bigskip
\begin{flushleft}
    {\setmainhangulfont[Path=FONT/]{KOPUBWORLD_DOTUM_PRO_BOLD.OTF}\textcolor{subheader}{{\LARGE\textbf{대각 행렬}}}}
\end{flushleft}

\begin{tcolorbox}[colback = white, colframe = Definition, title = \textmd{정의: 대각 행렬}]
    주대각 원소를 제외한 나머지 원소가 모두 $0$인 행렬을 대각 행렬이라고 한다. 
    \[ \begin{pmatrix} 2 & 0 \\ 0 & 3 \end{pmatrix}, \ 
    \begin{pmatrix} 2 & 0 & 0 \\ 0 & 3 & 0 \\ 0 & 0 & 4 \end{pmatrix} \]
\end{tcolorbox}


\bigskip
\begin{flushleft}
    {\setmainhangulfont[Path=FONT/]{KOPUBWORLD_DOTUM_PRO_BOLD.OTF}\textcolor{subheader}{{\LARGE\textbf{스칼라 행렬}}}}
\end{flushleft}

\begin{tcolorbox}[colback = white, colframe = Definition, title = \textmd{정의: 스칼라 행렬}]
    주대각 원소들이 모두 같은 값을 갖는 대각 행렬을 스칼라 행렬이라고 한다.
    \[ \begin{pmatrix} 3 & 0 \\ 0 & 3 \end{pmatrix}, \ 
    \begin{pmatrix} 5 & 0 & 0 \\ 0 & 5 & 0 \\ 0 & 0 & 5 \end{pmatrix} \]
\end{tcolorbox}

\newpage
\begin{flushleft}
    {\setmainhangulfont[Path=FONT/]{KOPUBWORLD_DOTUM_PRO_BOLD.OTF}\textcolor{subheader}{{\LARGE\textbf{전치 행렬}}}}
\end{flushleft}

\begin{tcolorbox}[colback = white, colframe = Definition, title = \textmd{정의: 전치 행렬}]
    행렬 $A$가 임의의 $m \times n$ 행렬일 때, $A$의 행과 열을 바꾼 $n \times m$ 행렬을 전치행렬이라고 한다. $A$의 전치행렬은 $A^T$라고 표현한다.
\end{tcolorbox}

\bigskip
\begin{flushleft}
    {\setmainhangulfont[Path=FONT/]{KOPUBWORLD_DOTUM_PRO_BOLD.OTF}\textcolor{subheader}{{\LARGE\textbf{대칭 행렬}}}}
\end{flushleft}

\begin{tcolorbox}[colback = white, colframe = Definition, title = \textmd{정의: 대칭 행렬}]
    어떤 정방 행렬이 자신과 자신의 전치 행렬이 같으면 그 행렬을 대칭 행렬이라고 한다. 행렬 $A$가 대칭 행렬이면 $A = A^T$를 만족한다.
\end{tcolorbox}

\bigskip
\begin{flushleft}
    {\setmainhangulfont[Path=FONT/]{KOPUBWORLD_DOTUM_PRO_BOLD.OTF}\textcolor{subheader}{{\LARGE\textbf{띠 행렬}}}}
\end{flushleft}

\begin{tcolorbox}[colback = white, colframe = Definition, title = \textmd{정의: 띠 행렬}]
    주대각 원소와 평행한 몇 줄의 원소들만 값을 갖고, 그 이외의 모든 원소들은 $0$인 행렬을 띠 행렬이라고 한다.
    \[ \begin{pmatrix} 5 & 4 & 0 \\ 6 & 7 & 2 \\ 0 & 9 & 3 \end{pmatrix}, 
    \begin{pmatrix} 5 & 4 & 0 & 0 \\ 0 & 7 & 2 & 0 \\ 0 & 0 & 3 & 8 \\ 0 & 0 & 0 & 9\end{pmatrix} \]
\end{tcolorbox}

\bigskip
\begin{flushleft}
    {\setmainhangulfont[Path=FONT/]{KOPUBWORLD_DOTUM_PRO_BOLD.OTF}\textcolor{subheader}{{\LARGE\textbf{삼각 행렬}}}}
\end{flushleft}

\begin{flushleft}
    상삼각 행렬이거나 하삼각 행렬을 간단하게 삼각 행렬이라고 합니다.
\end{flushleft}

\begin{tcolorbox}[colback = white, colframe = Definition, title = \textmd{정의: 상삼각 행렬}]
    주대각 원소의 아래쪽에 있는 모든 원소들이 $0$인 정방 행렬을 상삼각 행렬이라고 한다.
    \[ \begin{pmatrix} 3 & 5 \\ 0 & 4 \end{pmatrix}, \ 
    \begin{pmatrix} 5 & 6 & 7 \\ 0 & 8 & 9 \\ 0 & 0 & 3 \end{pmatrix} \]
\end{tcolorbox}
\begin{tcolorbox}[colback = white, colframe = Definition, title = \textmd{정의: 하삼각 행렬}]
    주대각 원소의 위쪽에 있는 모든 원소들이 $0$인 정방 행렬을 하삼각 행렬이라고 한다.
    \[ \begin{pmatrix} 3 & 0 \\ 5 & 4 \end{pmatrix}, \ 
    \begin{pmatrix} 5 & 0 & 0 \\ 6 & 7 & 0 \\ 8 & 9 & 3 \end{pmatrix} \]
\end{tcolorbox}

\newpage
\begin{flushleft}
    {\setmainhangulfont[Path=FONT/]{KOPUBWORLD_DOTUM_PRO_BOLD.OTF}\textcolor{subheader}{{\LARGE\textbf{역행렬}}}}
\end{flushleft}
\begin{tcolorbox}[colback = white, colframe = Definition, title = \textmd{정의: 역행렬}]
    임의의 정방 행렬 $A$에 대하여, $AB = BA = I$를 만족하는 정방 행렬 $B$가 존재할 때 $A$는 "가역"(역행렬이 존재함)이라고 하고, $B$는 $A$의 역행렬이라고 하며 $A^{-1}$로 표시한다.
\end{tcolorbox}
\begin{tcolorbox}[colback = white, colframe = Theorem, title = \textmd{정리: 2차 정방 행렬의 역행렬 }]
    2차 정방 행렬 $A = \begin{pmatrix} a & b\\ c & d  \end{pmatrix}$는 $\text{det}(A) = ad - bc \ne 0$이므로 가역이고, $A$의 역행렬은 다음과 같이 구한다.
    \[ A^{-1} = \frac{1}{ad - bc} \begin{pmatrix} d & -b \\ -c & a \end{pmatrix} \]
\end{tcolorbox}
\begin{flushleft}
    역행렬을 구하는 방법은 아주 다양합니다. 일반적으로 연립방정식을 이용하여 푸는데, 연립방정식을 푸는 방법은 가우스 소거법, 가우스-조단 방법 같이 직접적인 해를 구하는 방법과 야코비 방법, 가우스-자이델 방법 같이 반복적으로 해를 구하는 방법이 있습니다. 
    이에 대한 자세한 공부는 다음에 다루고 지금은 넘어가도록 하겠습니다.
\end{flushleft}

\bigskip
\begin{flushleft}
    {\setmainhangulfont[Path=FONT/]{KOPUBWORLD_DOTUM_PRO_BOLD.OTF}\textcolor{subheader}{{\LARGE\textbf{정칙 행렬}}}}
\end{flushleft}

\begin{tcolorbox}[colback = white, colframe = Definition, title = \textmd{정의: 정칙 행렬}]
    임의의 정방 행렬 $A$가 있을 때, $A$의 역행렬이 존재하는 행렬을 정칙 행렬이라고 한다. $\text{det} (A) \neq 0$이 되는 행렬을 말한다.    
    \[ \begin{pmatrix} 1 & 3 \\ 2 & 4 \end{pmatrix} \]
\end{tcolorbox}

\bigskip
\begin{flushleft}
    {\setmainhangulfont[Path=FONT/]{KOPUBWORLD_DOTUM_PRO_BOLD.OTF}\textcolor{subheader}{{\LARGE\textbf{직교 행렬}}}}
\end{flushleft}

\begin{tcolorbox}[colback = white, colframe = Definition, title = \textmd{정의: 직교 행렬}]
    임의의 행렬 $A$에 대하여, $A^{-1} = A^{T}$가 되는 행렬을 직교 행렬이라고 한다.
\end{tcolorbox}

\bigskip
\begin{flushleft}
    {\setmainhangulfont[Path=FONT/]{KOPUBWORLD_DOTUM_PRO_BOLD.OTF}\textcolor{subheader}{{\LARGE\textbf{특이 행렬}}}}
\end{flushleft}

\begin{tcolorbox}[colback = white, colframe = Definition, title = \textmd{정의: 특이 행렬}]
    임의의 정방 행렬 $A$가 있을 때, $A$의 역행렬이 존재하지 않는 행렬을 특이 행렬이라고 한다. $\text{det} (A) = 0$이 되는 행렬을 말한다.
    \[ \begin{pmatrix} 1 & 4 & 0 \\ 2 & 5 & 0 \\ 3 & 6 & 0 \end{pmatrix} \]
\end{tcolorbox}

\newpage
\begin{flushleft}
    {\setmainhangulfont[Path=FONT/]{KOPUBWORLD_DOTUM_PRO_BOLD.OTF}\textcolor{subheader}{{\LARGE\textbf{부울 행렬}}}}
\end{flushleft}

\begin{tcolorbox}[colback = white, colframe = Definition, title = \textmd{정의: 부울 행렬}]
    행렬의 원소들이 $0$이거나 $1$인 $m \times n$ 행렬을 부울 행렬이라고 한다.
\end{tcolorbox}

\begin{flushleft}
    부울 행렬의 연산과 부울 연산은 다른데, 부울 행렬에서 사용하는 연산자는 “접합”, “교합”, “부울곱”입니다.
\end{flushleft}

\begin{tcolorbox}[colback = white, colframe = Definition, title = \textmd{정의: 접합}]
    $A = (a_{ij}), \ B = (b_{ij})$인 $m \times n$ 부울 행렬이 존재할 때,
    $A \vee B = C = (C_{ij})$를 $A$와 $B$의 접합이라고 하며, 다음과 같이 정의한다.
    \[ (c_{ij}) =  
    \begin{cases}
        1, \quad a_{ij}= 1 \text{ 혹은 } b_{ij} = 1 \ 인 \ 경우  \\
        0, \quad a_{ij} \ 와 \ b_{ij} \ 가 \ 모두 \ 0 \ 인 \ 경우 \\
    \end{cases} \]
\end{tcolorbox}

\begin{tcolorbox}[colback = white, colframe = Definition, title = \textmd{정의: 교합}]
    $A = (a_{ij}), \ B = (b_{ij})$인 $m \times n$ 부울 행렬이 존재할 때,
    $A \wedge B = D = (d_{ij})$를 $A$와 $B$의 교합이라고 하며, 다음과 같이 정의한다.
    \[ (d_{ij}) =  
    \begin{cases}
        1, \quad a_{ij} \ 와 \ b_{ij} \ 가 \ 모두 \ 1 \ 인 \ 경우 \\
        0, \quad a_{ij}= 0 \text{ 혹은 } b_{ij} = 0 \ 인 \ 경우
    \end{cases} \]
\end{tcolorbox}

\begin{tcolorbox}[colback = white, colframe = Definition, title = \textmd{정의: 부울곱}]
    $E \odot F = G = (g_{ij})$를 $E$와 $F$의 부울곱이라고 한다. 행렬 $E = (e_{ik})$가 $m \times p$ 부울 행렬이고 
    $F = (f_{ij})$가 $p \times n$ 부울 행렬이면 $E$와 $F$의 부울곱은 $m\times n$ 부울 행렬이 되고 다음과 같이 정의한다.
    \[ (g_{ij}) =  
    \begin{cases}
        1, \quad 어떤 \ k(1 \leq k \leq p) \ 에 \ 대해  \ a_{ij}= 1,\ b_{ij} = 1 \ 인 \ 경우  \\
        0, \quad 그 \ 외에
    \end{cases} \]
\end{tcolorbox}

\begin{tcolorbox}[colback = white, colframe = Theorem, title = \textmd{정의: 부울곱}]
    $A, \ B, \ C$가 부울 행렬일 때, 다음 성질이 성립한다.
    \begin{itemize}
        \item $A \vee B = B \vee A \\ A \wedge B = B \wedge A  $ (교환법칙)
        \item $(A \vee B ) \vee C = A \vee (B\vee C) \\ (A \wedge B ) \wedge C = A \wedge (B\wedge C)$ (결합법칙)
        \item $A \wedge (B\vee C) = (A \wedge B) \vee (A \wedge C) \\ A \vee (B\wedge C) = (A \vee B) \wedge (A \vee C) $ (분배법칙)
        \item $A \odot (B \odot C)= (A \odot B) \odot C$ (결합법칙)
    \end{itemize}
\end{tcolorbox}

