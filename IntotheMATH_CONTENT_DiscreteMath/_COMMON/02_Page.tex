%%%%%%%%%%%%%%%%%%%%%%%%% 상단 제목 %%%%%%%%%%%%%%%%%%%%%%%%%
\begin{tikzpicture}[remember picture,overlay]
    \draw[fill = blue!40] (0.46,0) rectangle (25, 4); 
    \node[draw = blue!40] at (3, 1.7)  {\setmainhangulfont[Path=FONT/]{KOPUBWORLD_DOTUM_PRO_BOLD.OTF}{\textcolor{white}{{\huge 이산수학 개론}}}};  
    \node[draw = blue!40] at (5.3, 0.8)  {\setmainhangulfont[Path=FONT/]{KOPUBWORLD_DOTUM_PRO_MEDIUM.OTF}{\textcolor{white}{{\small 행과 열을 수학적으로 표기하는 기법인 행렬 대해서 알아봅시다.}}}};
    % (x축, y축)
\end{tikzpicture}

\begin{tikzpicture}[remember picture,overlay]
    \draw[fill = black!90] (-2.51, 3.4) rectangle (0.45, 0.55); 
    \node[] at (-1, 2.33) {\setmainfont[Path=FONT/]{KOPUBWORLD_BATANG_PRO_MEDIUM.OTF}{\textcolor{skyblue}{{\Huge {Alpha}}}}};
    \node[] at (-1, 1.45)  {\setmainfont[Path=FONT/]{KOPUBWORLD_BATANG_PRO_BOLD.OTF}{{\textcolor{skyblue}{{\Huge\textbf {1.1.2}}}}}};
\end{tikzpicture}

%%%%%%%%%%%%%%%%%%%%%%%%% 내용 %%%%%%%%%%%%%%%%%%%%%%%%%
\begin{flushleft}
    {\setmainfont[Path=FONT/]{KOPUBWORLD_DOTUM_PRO_BOLD.OTF} {\textcolor{header}{{\huge\textbf{1.}}}}}
    {\setmainhangulfont[Path=FONT/]{KOPUBWORLD_DOTUM_PRO_BOLD.OTF} {\textcolor{header}{{\huge\textbf{행렬}}}}}
\end{flushleft}

\begin{flushleft}
    행렬은 행과 열로 나열하는 것을 말합니다. 기본적으로 연립방정식을 풀기 위하여 만들어진 개념이며, 수, 문자, 함수 등을 괄호 안에 배열한 것입니다. 행렬의 각 성분은 실수여야 하고, 이는 “스칼라(Scalar)”라고 합니다. 스칼라는 크기만 있고 방향을 가지지 않는 양을 말하며, 벡터와는 반대되는 개념입니다. 수학적으로 정의한 행렬은 다음과 같습니다. 
\end{flushleft}

\begin{tcolorbox}[colback = white, colframe = Definition, title = \textmd{정의: 행렬}]
    행렬 $A$는 실수들을 사각형의 배열로 표시한 것이다(단, $m$과 $n$은 양의 정수). 각각 $n$쌍으로 된 $m$개의 수평 성분 ($a_{i1} \ a_{i2} \ a_{i3} \cdots  a_{in} $) (단, $1 \leq i \leq m $)을 $A$의 행(row)이라고 하고, 각각 $m$쌍으로 된 $n$개의 수직 성분 ($a_{1j} \ a_{2j} \ a_{3j} \cdots  a_{mj} $)(단, $1 \leq j \leq n$)을 $A$의 열(column)이라고 한다.
    $$ A = \begin{pmatrix} a_{11} & a_{12} & a_{13} & \cdots & a_{1n} \\  
a_{21} & a_{22} & a_{23} & \cdots  & a_{2n}  \\ && \vdots &&  \\
a_{m1} & a_{m2} & a_{m3} & \cdots & a_{mn}\end{pmatrix}  $$
\end{tcolorbox}

\begin{flushleft}
행렬 $A$를 $A = (a_{ij})$로 표시하는데, 이때, 원소 $a_{ij}$는 $i$행의 $j$번째 열의 원소를 나타냅니다. $A$를 $m \times n$ 행렬이라고 하며, $m \ by \ n$ 행렬로 읽습니다.
\end{flushleft}
