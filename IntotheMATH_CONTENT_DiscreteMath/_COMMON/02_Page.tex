%%%%%%%%%%%%%%%%%%%%%%%%% 상단 제목 %%%%%%%%%%%%%%%%%%%%%%%%%
\begin{tikzpicture}[remember picture,overlay]
    \draw[fill = blue!40] (0.46,0) rectangle (25, 4); 
    \node[draw = blue!40] at (3, 1.7)  {\setmainhangulfont[Path=FONT/]{KOPUBWORLD_DOTUM_PRO_BOLD.OTF}{\textcolor{white}{{\huge 이산수학 개론}}}};  
    \node[draw = blue!40] at (5.3, 0.8)  {\setmainhangulfont[Path=FONT/]{KOPUBWORLD_DOTUM_PRO_MEDIUM.OTF}{\textcolor{white}{{\small 행과 열을 수학적으로 표기하는 기법인 행렬 대해서 알아봅시다.}}}};
    % (x축, y축)
\end{tikzpicture}

\begin{tikzpicture}[remember picture,overlay]
    \draw[fill = black!90] (-2.51, 3.4) rectangle (0.45, 0.55); 
    \node[] at (-1, 2.33) {\setmainfont[Path=FONT/]{KOPUBWORLD_BATANG_PRO_MEDIUM.OTF}{\textcolor{skyblue}{{\Huge {Alpha}}}}};
    \node[] at (-1, 1.45)  {\setmainfont[Path=FONT/]{KOPUBWORLD_BATANG_PRO_BOLD.OTF}{{\textcolor{skyblue}{{\Huge\textbf {1.1.2}}}}}};
\end{tikzpicture}

%%%%%%%%%%%%%%%%%%%%%%%%%%%%%%%%%%%%%%%%%%%%%%%%%% 행렬 %%%%%%%%%%%%%%%%%%%%%%%%%%%%%%%%%%%%%%%%%%%%%%%%%%

\begin{flushleft}
    {\setmainfont[Path=FONT/]{KOPUBWORLD_DOTUM_PRO_BOLD.OTF} {\textcolor{header}{{\huge\textbf{1.}}}}}
    {\setmainhangulfont[Path=FONT/]{KOPUBWORLD_DOTUM_PRO_BOLD.OTF} {\textcolor{header}{{\huge\textbf{행렬}}}}}
\end{flushleft}

\begin{flushleft}
    행렬은 행과 열로 나열하는 것을 말합니다. 기본적으로 연립방정식을 풀기 위하여 만들어진 개념이며, 수, 문자, 함수 등을 괄호 안에 배열한 것입니다. 행렬의 각 성분은 실수여야 하고, 이는 “스칼라(Scalar)”라고 합니다. 스칼라는 크기만 있고 방향을 가지지 않는 양을 말하며, 벡터와는 반대되는 개념입니다. 수학적으로 정의한 행렬은 다음과 같습니다. 
\end{flushleft}

\begin{tcolorbox}[colback = white, colframe = Definition, title = \textmd{정의: 행렬}]
    행렬 $A$는 실수들을 사각형의 배열로 표시한 것이다(단, $m$과 $n$은 양의 정수). 각각 $n$쌍으로 된 $m$개의 수평 성분 ($a_{i1} \ a_{i2} \ a_{i3} \cdots  a_{in} $) (단, $1 \leq i \leq m $)을 $A$의 행(row)이라고 하고, 각각 $m$쌍으로 된 $n$개의 수직 성분 ($a_{1j} \ a_{2j} \ a_{3j} \cdots  a_{mj} $)(단, $1 \leq j \leq n$)을 $A$의 열(column)이라고 한다.
    \[ A = \begin{pmatrix} a_{11} & a_{12} & a_{13} & \cdots & a_{1n} \\  
        a_{21} & a_{22} & a_{23} & \cdots  & a_{2n}  \\ && \vdots &&  \\
        a_{m1} & a_{m2} & a_{m3} & \cdots & a_{mn}\end{pmatrix}  \]
\end{tcolorbox}

\begin{flushleft}
행렬 $A$를 $A = (a_{ij})$로 표시하는데, 이때, 원소 $a_{ij}$는 $i$행의 $j$번째 열의 원소를 나타냅니다. $A$를 $m \times n$ 행렬이라고 하며, $m \ by \ n$ 행렬로 읽습니다.
\end{flushleft}

\bigskip
\begin{flushleft}
    {\setmainfont[Path=FONT/]{KOPUBWORLD_DOTUM_PRO_BOLD.OTF} {\textcolor{header}{{\huge\textbf{2.}}}}}
    {\setmainhangulfont[Path=FONT/]{KOPUBWORLD_DOTUM_PRO_BOLD.OTF} {\textcolor{header}{{\huge\textbf{행렬의 연산}}}}}
\end{flushleft}

\begin{flushleft}
    기초적인 행렬의 연산을 알아보기 전, "영행렬"과 "행렬의 같음 성질"에 대해서 알아야 합니다.
    먼저, 영행렬부터 알아보도록 하겠습니다.
\end{flushleft}

\bigskip
\begin{flushleft}
    {\setmainhangulfont[Path=FONT/]{KOPUBWORLD_DOTUM_PRO_BOLD.OTF}\textcolor{subheader}{{\LARGE\textbf{영행렬}}}}
\end{flushleft}

\begin{flushleft}
    만약 각 성분이 모두 0이라면, 이 행렬을 “영행렬”이라고 하고 0으로 표시합니다. 
\end{flushleft}

\begin{tcolorbox}[colback = white, colframe = Definition, title = \textmd{정의: 영행렬}]
    각 성분이 0인 행렬을 말한다. 영행렬은 $0$으로 표시한다.
    \[ \begin{pmatrix} 0 & 0 \\ 0 & 0 \end{pmatrix}, \ 
    \begin{pmatrix} 0 &0 & 0 \\ 0 & 0 & 0 \end{pmatrix}, \ 
    \begin{pmatrix} 0 & 0 & 0 \\ 0 & 0 & 0 \\ 0 & 0 & 0 \end{pmatrix} \]
\end{tcolorbox}

\bigskip
\begin{flushleft}
    {\setmainhangulfont[Path=FONT/]{KOPUBWORLD_DOTUM_PRO_BOLD.OTF}\textcolor{subheader}{{\LARGE\textbf{두 행렬의 같음}}}}
\end{flushleft}

\begin{flushleft}
    두 행렬이 있을 때, 그 두 행렬이 같을 조건은 다음과 같습니다.
\end{flushleft}

\begin{tcolorbox}[colback = white, colframe = Definition, title = \textmd{정의: 두 행렬의 같음}]
    두 행령이 $m \times n$ 행렬이고 대응하는 원소가 모두 같으면 $A = (a_{ij})$와 $B = (b_{ij})$는 같다라고 하며, $A = B$ 라고 표현합니다.
\end{tcolorbox}

\newpage

\begin{flushleft}
    지금부터는 본격적인 행렬의 연산 방식에 대해 알아보도록 하겠습니다. 행렬의 기초적인 연산으로는 행렬의 덧셈, 행렬의 실수곱, 행렬곱 등이 있습니다.
\end{flushleft}

% \bigskip
\begin{flushleft}
    {\setmainhangulfont[Path=FONT/]{KOPUBWORLD_DOTUM_PRO_BOLD.OTF}\textcolor{subheader}{{\LARGE\textbf{행렬의 합}}}}
\end{flushleft}

\begin{flushleft}
    행렬의 합은 다음과 같이 정의합니다.
\end{flushleft}

\begin{tcolorbox}[colback = white, colframe = Definition, title = \textmd{정의: 행렬의 합}]
    행렬 $A$와 $B$가 같은 크기와 행과 열을 가지면, 행렬 $A$와 $B$의 행렬의 합은 $A + B$로 표시하고 $A + B = (a_{ij}) + (b_{ij}) = (a_{ij}+b_{ij})$이다. 즉, 대응하는 성분끼리 합을 구하면 그 값이 $A + B$의 각 성분이 되는 것이다.
\end{tcolorbox}

\begin{flushleft}
    행렬의 합을 통하여 알 수 있는 행렬의 덧셈 성질은 다음과 같습니다.
\end{flushleft}

\begin{tcolorbox}[colback = white, colframe = Theorem, title = \textmd{정리: 행렬의 덧셈 성질}]
    \begin{enumerate}
        \item $A + B = B + A$    (덧셈의 교환법칙)
        \item $(A + B) + C = A + (B + C)$    (덧셈의 결합법칙)
        \item $A + 0 = 0 + A = A$    (덧셈의 항등법칙)
    \end{enumerate}
\end{tcolorbox}

\bigskip
\begin{flushleft}
    {\setmainhangulfont[Path=FONT/]{KOPUBWORLD_DOTUM_PRO_BOLD.OTF}\textcolor{subheader}{{\LARGE\textbf{스칼라 곱과 행렬의 곱}}}}
\end{flushleft}

\begin{flushleft}
    스칼라 곱은 행렬에 실수배를 하는 것입니다. 스칼라 곱은 다음과 같이 정의됩니다.
\end{flushleft}

\begin{tcolorbox}[colback = white, colframe = Definition, title = \textmd{정의: 스칼라 곱}]
    $A$가 $m \times n$ 행렬이고 $c$가 실수이면 다음이 성립합니다.
    $$ c \cdot A = c \cdot (a_{ij}) = (c \times a_{ij}) $$  
\end{tcolorbox}

\begin{flushleft}
    두 행렬을 곱하는 행렬의 곱은 다음과 같이 정의됩니다.
\end{flushleft}

\begin{tcolorbox}[colback = white, colframe = Definition, title = \textmd{정의: 행렬의 곱}]
    $A = (a_{ij})$는 $m \times p $ 행렬이고 $B = (b_{ij})$는 $p \times n$ 행렬일 때, 행렬의 곱 $AB$는 $m \times n $ 행렬이고 $C = (c_{ij})$가 된다. 이때, 
    \[ C_{ij} = a_{i1}b_{1j} + a_{i2}b_{2j} \cdots + a_{ip}b_{pj} \quad (\text{단, } 1 \leq i \leq m, \ 1 \leq j \leq n) \]
    입니다. 즉, 행렬의 곱 $AB$는 다음과 같이 구할 수 있습니다. \\

    $ AB = \begin{pmatrix} a_{11} & a_{12} & \cdots & a_{1p} \\  
    a_{21} & a_{22} & \cdots  & a_{2p}  \\ && \vdots  \\ 
    a_{i1} & a_{i2} & \cdots & a_{ip}
    \\  && \vdots   \\ 
    a_{m1} & a_{m2} & \cdots & a_{mp}\end{pmatrix} $
    $ \begin{pmatrix} b_{11} & b_{12} &  \cdots b_{1j} & \cdots & b_{1n} \\  
    b_{21} & b_{22} & \cdots b_{1j} & \cdots & b_{2n}  \\ && \vdots &&  \\ && \vdots &&  \\ && \vdots &&  \\
    b_{p1} & b_{p2} & \cdots b_{1j} & \cdots & b_{pn}\end{pmatrix} $
    =
    $ \begin{pmatrix} c_{11} & c_{12} & c_{13} & \cdots & c_{1n} \\  
    c_{21} & c_{22} & c_{23} & \cdots  & c_{2n}  \\ && \vdots &&  \\ 
    & \cdots & c_{ij} & \cdots 
    \\ && \vdots &&  \\ 
    c_{m1} & c_{m2} & c_{m3} & \cdots & c_{mn}\end{pmatrix} $
\end{tcolorbox}

